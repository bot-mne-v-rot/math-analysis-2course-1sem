\chapter{Теория меры}
\section{Системы множеств}
\underline{Небольшое введение в теорию меры:} на плоскости у нас есть понятие площади. На 
различных множествах оно может быть задано по разному и может давать разные результаты. Мы хотим сузить класс множеств, на которых 
зададим понятие меры, чтобы результаты были одинаковыми. Также мы хотим обобщить понятие интеграла. Сейчас класс интегрируемых функций у нас
слишком мал. 

Пусть есть множество $X$. Будем говорить про подмножества $X$. Возьмем некое семейство подмножеств $\A \subset 2^X$. Также введем обозначение:

\begin{conj}
  За $A \sqcup B$ обозначим \textbf{дизъюнктивное объединение}. Это объединение $A$ и $B$, при условии, что они не пересекаются. 
  В данном случае $A$ и $B$ называются дизъюнктными подмножествами.
\end{conj}

\begin{conj}
  Пусть есть множество $E = \bigsqcup\limits_{\alpha \in I} E_\alpha$. Тогда это самое $\bigsqcup\limits_{\alpha \in I} E_\alpha$ называется 
  \textbf{разбиением} множества $E$
\end{conj}

\underline{Напоминание}: 
\begin{gather*}
  X \setminus \bigcup\limits_{\alpha \in I} A_\alpha = \bigcap\limits_{\alpha \in I} X \setminus A_\alpha \\
  X \setminus \bigcap\limits_{\alpha \in I} A_\alpha = \bigcup\limits_{\alpha \in I} X \setminus A_\alpha
\end{gather*}

Свойства системы множеств $\A$:
\begin{enumerate}
  \item[$(\sigma_0)$:] $A, B \in \A \Longrightarrow A \cup B \in A$
  \item[$(\delta_0)$:] $A, B \in \A \Longrightarrow A \cap B \in A$
  \item[$(\sigma)$:] $A_1, A_2, A_3, \dots \in \A \Longrightarrow \bigcup\limits_{n=1}^{\infty} A_n \in \A$
  \item[$(\delta)$:] $A_1, A_2, A_3, \dots \in \A \Longrightarrow \bigcap\limits_{n=1}^{\infty} A_n \in \A$
\end{enumerate}

\begin{conj}
  $\A$ -- \textbf{симметричная} система, если:
  \begin{gather*}
    A \in \A \Longrightarrow X \setminus A \in \A 
  \end{gather*}
\end{conj}

\textbf{Утверждение.} 

Если $\A$ -- симметрична, то $(\sigma_0) \Longleftrightarrow (\delta_0)$ и $(\sigma) \Longleftrightarrow (\delta)$

\begin{proof}
  \begin{gather*}
    X \setminus (A \cup B) = (X \setminus A) \cap (X \setminus B) \\
    (A \cup B) \in \A \Longrightarrow (X \setminus A) \cap (X \setminus B) \in \A \\
    (\sigma_0) \Longrightarrow (\delta_0)
  \end{gather*}
  Обратно аналогично, второе утверждение аналогично.
\end{proof}

\begin{conj}
  $\A$ -- алгебра, если: $\varnothing \in \A$, $\A$ симметрична и выполняются свойства $(\sigma_0)$ и $(\delta_0)$.
\end{conj}

\begin{conj}
  $\A$ -- $\sigma$-алгебра, если: $\varnothing \in \A$, $\A$ симметрична и выполняются свойства $(\sigma)$ и $(\delta)$.
\end{conj}

Свойства алгебры:
\begin{enumerate}
  \item $\varnothing, X \in \A$
  \item $A, B \in \A \Longrightarrow A \setminus B \in \A$
  \begin{proof}
    $A \setminus B = A \cap (X \setminus B)$. $X \setminus B \in \A$ по симметричености, тогда $A \cap (X \setminus B) \in \A$ по $(\delta_0)$. А тогда $A \setminus B \in \A$
  \end{proof}
  \item Если $A_1, \dots, A_n \in \A$, то $\bigcup\limits_{k=1}^n A_k$ и $\bigcap\limits_{k=1}^n A_k \in \A$
  \begin{proof}
    Индукция по $n$
  \end{proof}
\end{enumerate}

\textbf{Примеры}:
\begin{enumerate}
  \item $X = \R^n$ \\
  $\A = \{$все ограниченные множества и их дополнения$\}$ -- алгебра, но не $\sigma$-алгебра.
  \item $2^X$ -- $\sigma$-алгебра
  \item $X \supset Y \qquad \A$ -- алгебра ($\sigma$-алгебра) подмножеств $X$ \\
  $\B := \{ A \cap Y : A \in \A \}$ -- алгебра ($\sigma$-алгебра) подмножеств $Y$
  \begin{proof}
    Хотим проверить симметричность. Для некоторого $B \in \B$:
    \begin{align*}
      B &= A \cap Y & Y \setminus B &= Y \setminus A \\
        &           &               &= Y \cap \subsetbelow{(X \setminus A)}{A} \\
        &           &               &\Rightarrow Y \setminus B \in B
    \end{align*}
    Объединение тоже есть: $A_1, A_2 \in X \Longrightarrow A_1 \cap A_2 \in X$
  \end{proof} 
  $\B$ называется индуцированной алгеброй($\sigma$-алгеброй)
  \item $\A_\alpha$ -- алгебры($\sigma$-алгебры) подмножеств в $X$. Тогда $\bigcap\limits_{\alpha \in I} \A_\alpha$ -- алгебра($\sigma$-алгебра) подмножеств в $X$.
\end{enumerate}

Встал вопрос. Если взять два множества из $X$, как будет выглядеть минимальная алгебра, которая содержет их оба? 

\begin{theorem}
  Пусть $\E$ -- система подмножеств $X$. Тогда существует наименьшая по включению $\sigma$-алгебра, содержащая $\E$. Под минимальной по включению 
  имеется в виду, что эта алгебра не содержит других, но при этом все остальные содержат ее.
\end{theorem}

\begin{proof}
  Пусть $\A_\alpha$ -- всевозможные $\sigma$-алгебры, которые содержат $\E$. $2^X$ -- алгебра и $2^X \supset \E$ -- то есть алгебры, указанные в теоремы 
  как минимум существуют.

  Теперь пересечем все такие $\sigma$-алгебры:
  \begin{gather*}
    \B := \bigcap\limits_{\alpha \in I} \A \text{ -- $\sigma$-алгебра}
  \end{gather*}
  Утверждается, что $\B$ -- искомая $\sigma$-алгебра. Пойдем от обратного. Пусть есть $\sigma$-алгебра $\mathcal{C} \supset \E$ и $\mathcal{C} \not\supset \B$.
  Но такого не может быть, так как $\mathcal{C} \supset \E \Longrightarrow \mathcal{C} = \A_0 \Longrightarrow \mathcal{C} \supset \B$
\end{proof}

\begin{conj}
  Наша $\sigma$-алгебра, которую мы получили и которая содержет $\E$ -- это \textbf{борелевская} оболочка $\E$. Обозначается как $\B(\E)$
\end{conj}

\begin{conj}
  Борелевская оболочка системы всех открытых множеств -- \textbf{борелевская $\sigma$-алгебра}. Если $X = \R^n$, обозначается как $\B^n$
\end{conj}

\begin{notice}
  \begin{gather*}
    \underset{\text{континуум}}{\B^n} \neq \underset{\text{не континуум}}{2^{\R^n}}
  \end{gather*}
\end{notice}

\begin{conj}
  $\RR \supset 2^X$ -- кольцо, если:
  \begin{itemize}
    \item $\varnothing \in \RR$
    \item Если $A, B \in \RR$, то $A \cap B \in \RR, A \setminus B \in \RR$ и $A \cap B \in \RR$ 
  \end{itemize}
\end{conj}

\begin{conj}
  $\PP \subset 2^X$ -- полукольцо, если:
  \begin{itemize}
    \item $\varnothing \in \PP$
    \item $A, B \in \PP \Longrightarrow A \cap B \in \PP$
    \item $A, B \in \PP \Longrightarrow \exists Q_1, Q_2, \dots, Q_n \in \PP$, такие, что:
    \begin{gather*}
      A \setminus B = \bigsqcup\limits_{k=1}^n Q_k
    \end{gather*}
  \end{itemize}
\end{conj}

Вопрос справедливый, зачем нам вообще такая кривая структура как полукольцо? Ладно еще просто кольцо, но это уже вообще абсурд. Разгадка в примере, 
для примерного описания свойств которого специально слепили структуру. 

\textbf{Пример}: в $\R$ рассмотрим различные полуинтервалы:
\begin{gather*}
  \PP = \{ [a, b) : a, b \in \R \} \text{ -- полукольцо} 
\end{gather*} 
Если пробовать делать то же самое в $\R^2$, то получим:
\begin{gather*}
  \PP = \{ [a, b) \times [c, d) : a, b, c, d \in \R \} \text{ -- полукольцо} 
\end{gather*}

\begin{lemma}
  $A_n \subset X$. Тогда:
  \begin{gather*}
    \bigcup\limits_{k=1}^n A_k = \bigsqcup\limits_{k=1}^n \left( A_k \setminus \bigcup\limits_{i=1}^{k-1} A_i \right) \\
    \text{и аналогично для счетного множества} \\
    \bigcup\limits_{k=1}^\infty A_k = \bigsqcup\limits_{k=1}^\infty \left( A_k \setminus \bigcup\limits_{i=1}^{k-1} A_i \right) 
  \end{gather*}
\end{lemma}

\begin{proof}
  \begin{gather*}
    \text{Обозначим } B_k := A_k \setminus \bigcup_{i=1}^{k-1} A_i
  \end{gather*}
  $B_k \subset A_k$. Сначала докажем, что $B$-шки не пересекаются: пусть $j < k$. Надо проверить, что 
  $B_j \cap B_k = \varnothing$. 
  \begin{gather*}
    \subsetbelow{B_j}{A_j} \qquad \cap \stackbelow{B_k}{A_k \setminus (A_1 \cup A_2 \cup \dots \cup \text{\circled{$A_j$}} \cup \dots \cup A_{k-1})}
  \end{gather*} 
  Так как из $B_k$ выкуинули $A_j$, а $B_j$ лежит в $A_j$, то из $B_k$ выкинуто и $B_j$, то есть $B_j \cap B_k = \varnothing$. 
  
  Теперь докажем, что объединения $A$-шек и $B$-шек равны. 
  \begin{enumerate}
    \item[``$\supset$'':] Так как $A_k \supset B_k$
    \item[``$\subset$'':] Возьмем $x$ лежащий в левой части. Тогда $\exists k$, такое, что 
    $x \in A_k$. Возьмем наименьший индекс $k_0$, такой, что $x \in A_{k_0}$, и при этом $x \notin A_i$ при $i < k_0$.
    Тогда:
    \begin{gather*}
      x \in B_{k_0} = \underbrace{A_{k_0}}_{\text{тут есть } x} \setminus \underbrace{(A_1 \cup \dots \cup A_{k_0 - 1})}_{\text{тут нет } x}
    \end{gather*}
    То есть какой бы мы ни взяли элемент в левой части, он обязан будет лежать в правой.
  \end{enumerate}
\end{proof}

\begin{theorem}
  $\PP$ -- полукольцо. Тогда $P_1, P_2, \dots, P_n \in \PP$ и:
  \begin{enumerate}
    \item $P\setminus \bigcup\limits_{k=1}^n P_k = \bigsqcup\limits_{j=1}^m Q_j$ для некоторых $Q_1, \dots, Q_m \in \PP$
    \item $\bigcup\limits_{k=1}^n P_k = \bigsqcup\limits_{k=1}^n \bigsqcup\limits_{j=1}^{m_k} Q_{k_j}$, где $Q_{k_j} \in \PP$ и $Q_{k_j} \subset P_k$
    \item $\bigcup\limits_{k=1}^\infty P_k = \bigsqcup\limits_{k=1}^\infty \bigsqcup\limits_{j=1}^{m_k} Q_{k_j}$, где $Q_{k_j} \in \PP$ и $Q_{k_j} \subset P_k$
  \end{enumerate}
\end{theorem}

\begin{proof} \quad 

  \begin{enumerate}
    \item Будем доказывать по индукции. База: $n=1$ -- определение полукольца. Переход $n \longrightarrow n+1$:
    \begin{gather*}
      P \setminus \bigcup\limits_{k=1}^{n+1} P_k = \left( P \setminus \bigcup\limits_{k=1}^n P_k\right) \setminus P_{n+1} \xlongequal{\text{инд. предп.}} \left( \bigsqcup\limits_{j=1}^m Q_j \right) \setminus P_{n+1} = \stackbelow{\underbrace{\bigsqcup\limits_{j=1}^m Q_j \setminus P_{n+1}}}{\oast}
    \end{gather*}
    $\oast = \bigsqcup\limits_{i=1}^{m_j} R_{j_i}$ по определению полукольца. 
    \item Доказывается сугубо применением леммы:
    \begin{gather*}
      \bigcup\limits_{k=1}^n P_k \xlongequal{\text{лемма}} \bigsqcup\limits_{k=1}^n \underset{\displaystyle\overset{\qquad\quad\;\;\displaystyle\parallel{\text{по п. 1}}}{\bigsqcup\limits_{i=1}^{m_k} Q_{k_i}}}{\underbrace{\left( P_k \setminus \bigcup\limits_{i=1}^{k-1} A_i \right)}} 
      = \bigsqcup\limits_{k=1}^n \bigsqcup\limits_{i=1}^{m_k} Q_{k_i}
    \end{gather*}
    \item То же самое, что и пункт 2, только не до $n$, а до $\infty$
  \end{enumerate}
\end{proof}

\begin{conj}
  $\A \subset 2^X, \B \subset 2^Y$. Зададим \textbf{декартово произведение} семейств множеств:
  \begin{gather*}
    \A \times \B = \{ \subsetbelow{A \times B}{X \times Y} : A \in \A, B \in \B \} \subset 2^{X \times Y} 
  \end{gather*} 
\end{conj}

\begin{theorem}
  Декартово произведение полуколец -- полукольцо.
\end{theorem}

\begin{proof}
  Пусть $\A$ и $\B$ -- полукольца. Будем проверять аксиомы полукольца у их декартова произведения:
  \begin{enumerate}
    \item $\varnothing \times \varnothing = \varnothing \in \A \times \B$ -- пустой есть
    \item Возьмем $A \times B \in \A \times \B$ и $A' \times B' \in \A \times \B$
    \begin{center}
      \begin{tikzpicture}
        \node[box1, fill=red, fill opacity=0.08] (c2) at (0,0) {};
        \node[box2, fill=blue, fill opacity=0.08] (c1) at (1.5,1.5) {};
        \draw[pattern={Lines[angle=-45,distance={3pt/sqrt(2)}]}, pattern color=blue] (0,0.5) rectangle (1.25,1.25);
        \node[] (A) at (0,-1.6) {$A$};
        \node[] (B) at (-1.6,0) {$B$};
        \node[] (A1) at (1.6,0.2) {$A'$};
        \node[] (B1) at (-0.3,1.6) {$B'$};
      \end{tikzpicture}
    \end{center}
    Тогда:
    \begin{gather*}
      (A \times B) \cap (A' \times B') = \inbelow{(A \cap A')}{\A} \times \inbelow{(B \cap B')}{\B} \subset \A \times \B
    \end{gather*}
    Все четенько 
    \item Осталось проверить последний пункт:
    \begin{center}
      \begin{tikzpicture}
        \node[box1, fill=white, fill opacity=0.08] (c2) at (0,0) {};
        \draw[pattern={Lines[angle=-45,distance={3pt/sqrt(2)}]}, pattern color=blue] (-1.25,-1.25) rectangle (1.25,1.25);
        \node[box2, fill=white, fill opacity=1] (c1) at (1.5,1.5) {};
        \node[box3, fill=white, fill opacity=0.08] (c2) at (0,0) {};
        \node[] (A) at (0,-1.6) {$A$};
        \node[] (B) at (-1.6,0) {$B$};
        \node[] (A1) at (1.6,0.2) {$A'$};
        \node[] (B1) at (-0.3,1.6) {$B'$};
      \end{tikzpicture}
    \end{center} 
  \end{enumerate}
  \begin{gather*}
    (A \times B) \setminus (A' \times B') = A \times (B \setminus B') \sqcup (A \setminus A') \times (B \cap B')
  \end{gather*}
  Сверяемся с картинкой, понимаем, что звучит как правда. Теперь:
  \begin{gather*}
    \stackbelow{\underbrace{A \times \stackbelow{\underbrace{(B \setminus B')}}{\bigsqcup\limits_{j=1}^n Q_j}}}{\bigsqcup\limits_{j=1}^n A \times Q_i} \sqcup \stackbelow{\underbrace{\stackbelow{\underbrace{(A \setminus A')}}{\bigsqcup\limits_{i=1}^n P_i} \times (B \cap B')}}{\bigsqcup\limits_{i=1}^m \bigsqcup\limits_{j=1}^n P_i \times (B \cap B')}
  \end{gather*} 
  Ну и вуаля. Вышло что-то дикое, но мы представили наше выражение в виде квадратного объединения кучи частей, как мы и хотели. 
\end{proof}

\begin{conj}
  Пусть $a, b \in \R^n$. Определим \textbf{замкнутый параллелипипед} следующим образом:
  \begin{gather*}
    [a, b] = [a_1, b_1] \times [a_2, b_2] \times \dots \times [a_n, b_n]
  \end{gather*}
  \textbf{Открытый параллелипипед} следующим образом:
  \begin{gather*}
    (a, b) = (a_1, b_1) \times (a_2, b_2) \times \dots \times (a_n, b_n)
  \end{gather*}
  А \textbf{ячейку} следующим образом:
  \begin{gather*}
    [a, b) = [a_1, b_1) \times [a_2, b_2) \times \dots \times [a_n, b_n)
  \end{gather*}
\end{conj}

\begin{theorem}
  Непустая ячейка -- это:
  \begin{enumerate}
    \item обьединение возрастающей последовательности замкнутых параллелипипедов 
    \item пересечение убывающей последовательности открытых параллелипипедов
  \end{enumerate}
\end{theorem}

\begin{proof} \quad 
  \begin{center}
    \begin{tikzpicture}
      \node[] (1) at (-1,1.5) {$1.$};
      \node[box4, fill=red, fill opacity=0.1] (c1) at (1.5,1.5) {};
      \node[box2, fill=white, fill opacity=1] (c1) at (1.43,1.43) {};

      \node[] (2) at (5.5,1.5) {$2.$};
      \node[box4, fill=blue, fill opacity=0.1] (c1) at (8,1.5) {};
      \node[box2, fill=white, fill opacity=1] (c1) at (8.07,1.57) {};
    \end{tikzpicture}
  \end{center}
  \begin{align*}
    [a, b) &= \bigcup\limits_{m=1}^\infty [a, b^{(m)}] & b^{(m)} &= \left( b_1 - \frac{1}{m}, b_2 - \frac{1}{m}, \dots, b_n - \frac{1}{m} \right) \\
    [a, b) &= \bigcap\limits_{m=1}^\infty [a^{(m)}, b] & a^{(m)} &= \left( a_1 - \frac{1}{m}, a_2 - \frac{1}{m}, \dots, a_n - \frac{1}{m} \right)
  \end{align*}
\end{proof}

\underline{Обозначение}: $\PP^n$ -- \textbf{полукольцо ячеек} в $\R^n$. А $\PP^n_{\Q}$ -- \textbf{полукольцо ячеек} с рациональными координатами в $\R^n$. 

\underline{Утверждение}: Это действительно полукольцо.

\begin{proof}
  Следует из предыдущей теоремы про декартово произведение.
\end{proof}

\begin{theorem}
  Всякое непустое открытое множество $G \in \R^n$ -- это счетное дизъюнктивное 
  объединение ячеек. Более того, можно считать, что $\Cl$ ячеек $\subset G$ и можно считать, что ячейки с рациональными координаты.
\end{theorem}

\begin{proof}
  Возьмем $x \in G$. 

  \begin{center}
    \begin{tikzpicture}[square/.style={draw=black, thick, rectangle, minimum height=0.95cm, minimum width=0.95cm},
      dashedsquare/.style={draw=black, dashed, rectangle, minimum height=0.8cm, minimum width=0.8cm}]
      \draw[thick, -, rounded corners=2mm] (0,0) \irregularcircle{2cm}{2mm};
      \draw[thick] (0,-1) circle(0.7);
      \fill[thick, black] (0,-1) circle(0.07) node[below right]{x};
      \node[square, fill=white, fill opacity=0] (c1) at (0,-1) {};
      \node[dashedsquare, fill=white, fill opacity=0] (c1) at (0,-1) {};
      \draw[white, thick, dashed] (-0.45, -0.525) -- (0.45, -0.525);
      \draw[white, thick, dashed] (0.474, -0.57) -- (0.474, -1.46);
    \end{tikzpicture}
  \end{center}
  Возьмем замкнутый шарик $\overline{B_r}(x) \subset G$. Впишем в шарик кубик и получим ячейку. немного отъедем внутрь и получим ячейку 
  $P_x$ с рациональными координатами, если они вдруг вышли иррациональными. Теперь:
  \begin{gather*}
    x \in P_x \subset \overline{B_r}(x) \subset G \Longrightarrow x \in \Cl P_x \subset \overline{B_r}(x) \subset G
  \end{gather*} 
  Итак если мы возьмем объединение ячеек по всем $x$ из $G$, мы получим $G$, то есть $\bigcup\limits_{x \in G} P_x = G$. Проверим это:
  \begin{itemize}
    \item[``$\subset$'':] Очевидно, так как $P_x \in G$
    \item[``$\supset$'':] Очевидно, так как $\forall x \in G$, $x$ пренадлежит некоторому $P_x$
  \end{itemize}
  Чтобы сделать объединение дизъюнктивным воспользуемся уже доказанной теоремой. 

  Остается проблема только со счетностью. Ячейки параметризуются кортежами из рациональных координат. Различных таких
  координат счетное количество. Выкинем из множества повторы и оно станет счетным.
\end{proof}

\follow 
\begin{gather*}
  \B (\PP^n) = \B (\PP_\Q^n) = \B^n
\end{gather*}

\begin{proof}
  Последовательно докажем включения
  \begin{enumerate}
    \item $\B (\PP^n) \supset \B (\PP_\Q^n)$:
    \begin{gather*}
      \PP^n \supset \PP^n_\Q \Longrightarrow \B (\PP^n) \supset \B (\PP_\Q^n)
    \end{gather*}
    \item $\B (\PP_\Q^n) \supset \B^n$:
    $\B(\PP_\Q^n)$ содержит все открытые множества (по теореме) $\Longrightarrow \B(\PP_\Q^n)$ содержит наименьшую $\sigma$-алгебру, натянутую на открытые множества, то есть содержит $\B^n$
    \item $\B^n \supset \B (\PP^n)$: $\B^n \supset \PP^n$: ячейка -- пересечение открытых параллелограммов
  \end{enumerate}
  \mybox[orange!15]{Сам не понял, что я тут написал в пруфе третьего включения. В тетради неразборчиво. Если забуду поправить, поправьте кто-нибудь.}
\end{proof}

\section{Объем и мера}

\begin{conj}
  $\PP$ -- полукольцо подмножеств $X$. И задана:
  \begin{gather*}
    \mu : \PP \longrightarrow [0, +\infty]
  \end{gather*}
  Тогда $\mu$ -- \textbf{объем}, если:
  \begin{enumerate}
    \item $\mu \varnothing = 0$
    \item Если $A_1, \dots, A_n \in \PP$ и $\bigsqcup\limits_{k=1}^n A_k \in \PP$, то:
    \begin{gather*}
      \mu \left( \bigsqcup\limits_{k=1}^n A_k \right) = \sum\limits_{k=1}^n \mu A_k
    \end{gather*}
  \end{enumerate}
\end{conj}

\begin{conj}
  $\mu$ -- \textbf{мера}, если:
  \begin{enumerate}
    \item $\mu \varnothing = 0$
    \item Если $A_1, A_2, \dots \in \PP$ и $\bigsqcup\limits_{k=1}^\infty A_k \in \PP$, то:
    \begin{gather*}
      \mu \left( \bigsqcup\limits_{k=1}^\infty A_k \right) = \sum\limits_{k=1}^\infty \mu A_k
    \end{gather*}
  \end{enumerate}
\end{conj}

\underline{Упражнение}: Доказать, что если $\mu \not\equiv +\infty$, то из $(2)$ следует $(1)$.

\textbf{Примеры:}
\begin{enumerate}
  \item Длина на полуинтервале в $\R$
  \item $g : \R \longrightarrow \R$ монотонно возрастает. При $a \leqslant b$:
  \begin{gather*}
    \nu_g [a, b) := g(b) - g(a) \qquad \text{ на полуинтевалах в } \R
  \end{gather*}
  \item \begin{align*}
    \lambda_n &: \PP^n \longrightarrow [0, +\infty) \\
    \lambda_n[a, b) &:= (b_1 - a_1)(b_2 - a_2) \dots (b_n - a_n) \\
    \lambda_n \varnothing &:= 0 
  \end{align*}
  \item $x_0 \in X, a > 0$
  \begin{gather*}
    \mu A := \begin{cases}
      0 \text{, если } x_0 \not \in A \\
      a \text{, если } x_0 \in A
    \end{cases}
  \end{gather*}
  Это объем
  \item Все ограниченные множества в $\R^n$ и их дополнения:
  \begin{gather*}
    \mu (\text{огр. множества}) = 0 \\
    \mu (\text{неогр. множества}) = 1
  \end{gather*}
  Это объем, но не мера
\end{enumerate}